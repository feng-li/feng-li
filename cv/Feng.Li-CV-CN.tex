\documentclass[twoside,a4paper,11pt]{article}

%% Fonts
\usepackage[fontset=adobe]{ctex}
\setmainfont{Garamond Premier Pro}
% \setmainfont{Adobe Garamond Pro}
\usepackage{fontawesome}

%% Geometry
\usepackage[twoside]{geometry}
\geometry{lmargin=0.5in,rmargin=0.5in,tmargin=1in,bmargin=1in}
\usepackage{setspace}
\setstretch{1.25}

%% BibLaTeX
\usepackage[backend=biber, style=authoryear, sorting=ydnt, dashed=none, maxcitenames=2, maxbibnames=99, giveninits=false, hyperref=true, alldates=year, natbib=true, uniquelist=false, uniquename=false]{biblatex}
\renewbibmacro{in:}{%
  \ifentrytype{article}{}{%
  \printtext{\bibstring{in}\intitlepunct}}}
\DeclareNameAlias{sortname}{given-family}

% Add annotation entry.
\renewbibmacro*{finentry}{%
  \iffieldundef{annotation}
    {\finentry}{\printfield{annotation} \finentry}}

% Remove unused entries
\AtEveryBibitem{%
  % \clearfield{addendum}%
  \iffieldundef{doi}{}{\clearfield{url}}% Use url if doi not exist
  \clearlist{language}%
  \clearfield{issn}%
  \clearfield{urlyear}%
  \clearfield{series}%
}

% replace and with &
\renewcommand*\finalnamedelim{\addspace\&\space}

% Number with brackets, bold italic for journal names
% volume (number) style
\renewbibmacro*{volume+number+eid}{%
  \printfield{volume}%
%  \setunit*{\adddot}% DELETED
  \setunit*{\addnbspace}% NEW (optional); there's also \addnbthinspace
  \printfield{number}%
  \setunit{\addcomma\space}%
  \printfield{eid}}
\DeclareFieldFormat[article]{number}{\mkbibparens{#1}}
\DeclareFieldFormat{journaltitle}{\mkbibbold{\mkbibitalic{#1}}}
\DeclareFieldFormat{booktitle}{\mkbibbold{\mkbibitalic{#1}}}

% Reverse enumerated numbers to bibliography and keep the default sorting order.
\usepackage{etaremune}
\defbibenvironment{bibliography}{\begin{etaremune}[itemsep=0ex,parsep=0pt]} {\end{etaremune}} {\item}
\renewcommand{\labelenumi}{[\theenumi]}

\addbibresource{publications-feng.bib}
%\addbibresource{references-feng.bib}

%% Highlight certain author name
\usepackage{xpatch}
\makeatletter
\newbibmacro*{name:bold}[2]{%
  \edef\blx@tmp@name{\expandonce#1, \expandonce#2}%
  \def\do##1{\ifdefstring{\blx@tmp@name}{##1}{\bfseries\listbreak}{}}%
  \dolistloop{\boldnames}}
\newcommand*{\boldnames}{}
\makeatother
\xpretobibmacro{name:family}{\begingroup\usebibmacro{name:bold}{#1}{#2}}{}{}
\xpretobibmacro{name:given-family}{\begingroup\usebibmacro{name:bold}{#1}{#2}}{}{}
\xpretobibmacro{name:family-given}{\begingroup\usebibmacro{name:bold}{#1}{#2}}{}{}
\xpretobibmacro{name:delim}{\begingroup\normalfont}{}{}
\xapptobibmacro{name:family}{\endgroup}{}{}
\xapptobibmacro{name:given-family}{\endgroup}{}{}
\xapptobibmacro{name:family-given}{\endgroup}{}{}
\xapptobibmacro{name:delim}{\endgroup}{}{}

% List your own name variants that are to be highlighted
\forcsvlist{\listadd\boldnames}
{{Li, Feng}, {Li, F.}, {李丰}}

%% Spacing
\usepackage{enumitem}
\setlist[enumerate,itemize]{nosep,listparindent=\parindent,leftmargin=2em}% Indented paragraphs within lists

\usepackage{ragged2e}
\justifying

% Adjust Section Styles
\usepackage[compact,raggedright]{titlesec}
\titlespacing{\section}{0pt}{0ex}{0ex}
\titlespacing{\subsection}{0pt}{0ex}{0ex}
\titlespacing{\subsubsection}{0pt}{0ex}{0ex}
\setlength{\parskip}{.0em}
\setlength{\parindent}{1em}

%% Disable section numbering
\setcounter{secnumdepth}{-1}

\usepackage{booktabs}
\usepackage{graphicx}
\usepackage[unicode=true]{hyperref}
\hypersetup{
  colorlinks=true,
  linkcolor=blue,
  citecolor=blue,
  urlcolor=blue}
\hypersetup
{
  pdfauthor={Feng Li, 李丰},
  pdfsubject={Feng Li, Curriculum Vitae; 李丰 个人简历},
  pdftitle={Feng Li, Curriculum Vitae; 李丰 个人简历},
  pdfkeywords={Feng Li, 李丰, Central University of Finance and Economics, Statistics, CV, Resume, http://feng.li/}
}

%% Fancy header and footer
\usepackage{fancyhdr}
\pagestyle{fancy}

\fancyhead[RO,LE]{李丰}
\fancyhead[C]{}
\fancyhead[LO,RE]{\emph{个人简历}}

\fancyfoot[C]{}
\fancyfoot[LO,RE]{}
\fancyfoot[RO,LE]{\vspace{0.5cm}\thepage}

\fancypagestyle{plain}{%
  \fancyhf{} % clear all header and footer fields
  \fancyfoot[LO,RE]{\vspace{0.5cm}\tiny\emph{\today 更新。最新个人简历请访问}  \url{https://feng.li/}}
  \fancyfoot[C]{}
  \fancyfoot[LE,RO]{\vspace{0.5cm}\thepage}
  \renewcommand{\headrulewidth}{0pt}
  \renewcommand{\footrulewidth}{0pt}}

\begin{document}
\justifying


\thispagestyle{plain}
\section{\Huge{李丰~~博士}}
\begin{center}
  \emph{\huge 个人简历}
\end{center}
\rule{\textwidth}{.01cm}

\section{个人信息}
\begin{tabular}{l l l l}
  姓名:   & 李丰 & 外文论文署名: & Feng Li                                 \\
  % 性别: & 男   & 出生日期:     & 1984年1月16日                           \\
  性别:   & 男   & 掌握语言:     & 中文(母语),英语(流利)              \\
\multicolumn{2}{l}{\includegraphics[height=1em]{orcid-logo}~~{\href{https://orcid.org/0000-0002-4248-9778}{0000-0002-4248-9778}}} & \multicolumn{2}{l}{\faGoogle~~  \url{https://scholar.google.com/citations?user=IN2QMXYAAAAJ}} \\
\end{tabular}

\section{工作任职}

\begin{tabular}{ll}
2024年7月—今 & 北京大学 光华管理学院 商务统计与经济计量系 副教授、研究员 博士生导师 \\
% 2024年7月—今 & 北京大学 光华管理学院 研究员 博士生导师 \\
\end{tabular}


\section{联系方式}

\begin{tabular}{ l l |  l  l l l}
  北京大学 光华管理学院 &  & \faEnvelope & \href{mailto:feng.li@gsm.pku.edu.cn}{feng.li@gsm.pku.edu.cn} \\
  商务统计与经济计量系   &  & \faHome     & \url{https://feng.li}                                        \\
  海淀区颐和园路5号      &  & \faGroup    & \url{https://kllab.org}                                      \\
  邮编:100871           &  & \faPhone    & +86 (0)10 6274 7602                                          \\
\end{tabular}



\section{工作经历}

\begin{tabular}{lll}
2020年11月—2024年6月 & 中央财经大学 统计与数学学院 & 副教授 \\
2013年9月—2020年10月 & 中央财经大学 统计与数学学院 & 讲师    \\
2016年7月—2022年12月 & 中央财经大学 统计与数学学院 & 副院长、数学教学部副主任
\end{tabular}


\section{教育背景}

\begin{tabular}{ l  p{0.75\textwidth}}
  2008 --- 2013 & \textbf{统计学~博士},瑞典斯德哥尔摩大学统计学系 \\
                & 博士论文:\emph{Bayesian Modeling of Conditional Densities}(\emph{贝叶斯条件密度建模}) \\
                & (获评2014年瑞典最佳统计学博士论文——Cramér Prize) \\
                & 导  师: Mattias Villani教授        \\
                & 答辩主席:Sylvia Frühwirth-Schnatter 教授 (奥地利维也纳经济大学,WU ) \\
  2007 --- 2008 & \textbf{统计学~硕士}, 瑞典达拉那大学统计学系 \\
  2003 --- 2007 & \textbf{统计学~本科}, 中国人民大学统计学院    \\
\end{tabular}

\section{研究兴趣}

\textbf{新技术驱动的预测方法 $\circ$  大数据分布式学习 $\circ$ 贝叶斯推断与统计计算}

\section{期刊任职}
\begin{itemize}
\item 国际期刊 \href{https://www.sciencedirect.com/journal/international-journal-of-forecasting}{International Journal of Forecasting} 副主编 (2025+)
\end{itemize}

\section{计算机技能}
\begin{itemize}

\item 擅长在 Spark 等现代大数据分布式计算平台设新颖的统计与预测算法。
\item 擅长在 AI CPU/GPU 集群进行大规模计算。
\item 熟练使用 Python, R, Scala, Bash 等主流计算机语言。
\end{itemize}


\section{纵向课题}

\begin{itemize}

\item 2025年 \textbf{国家自然科学基金重大项目}:\emph{大规模商务场景下的统计学习与管理实践},\textbf{子课题负责人},在研 。

\item 2022-2025年 \textbf{国家社会科学基金项目一般项目}:\emph{全局模型视角下的复杂分层经济预测研究},\textbf{项目负责人},结项(\textbf{结项鉴定优秀}) (20万元)。

\item 2020-2024年 \textbf{国家自然科学基金面上项目}:\emph{中医药临床疗效评价中基于目标值法的单臂临床研究方法体系的构建},\textbf{主要参与人}( 15万元)。

\item 2015-2018年 \textbf{国家自然科学基金青年项目}:\emph{贝叶斯柔性密度方法及其在高维金融数据中的应用},\textbf{项目负责人},结项(18万元)。

\item 2014-2016年 \textbf{教育部人文社科研究项目}:\emph{贝叶斯弹性高维密度方法在复杂数据的研究},\textbf{项目负责人},结项。

\end{itemize}

\section{横向课题}
\begin{itemize}

\item 2024-2025年 \textbf{北京香港马会:竞猜市场综合发展评价主题研究项目},项目负责人,结项(67万元)。

\item 2021-2023年 \textbf{阿里巴巴创新研究计划:电商场景下的复杂时间序列预测问题研究},项目负责人,结项(48万元)。

\end{itemize}

\section{代表性成果}
\begin{refsection}
\nocite{HuangY2024LocalInformation}
\nocite{ZhangB2023OptimalReconciliation}
\nocite{KangY2022ForecastForecasts}
\nocite{ZhuX2021LeastSquareApproximation}
% \nocite{LiX2020ForecastingTime}
\nocite{LiF2018ImprovingForecasting}
\printbibliography[heading=none]
\end{refsection}


\begin{refsection}
\section{学术发表}
\nocite{WangZ2026ImpactTransportation}
\nocite{PanZ2025MyopiaHigh}
\nocite{ZhongY2025OptimalStarting}
\nocite{WangWen2025VARX}
\nocite{GaoY2025GridPoint}
\nocite{WangH2024CatastropheDuration}
\nocite{HuangY2024LocalInformation}
\nocite{RenY2023InfiniteForecast}
\nocite{ZhangG2023ProbabilisticForecast}
\nocite{LiF2024ForecasterReview}
\nocite{LiL2023ForecastingLarge}
\nocite{ZhangB2023OptimalReconciliation}
\nocite{LiL2023FeaturebasedIntermittent}
\nocite{WangX2023ForecastCombinations}
\nocite{PanR2022NoteDistributed}
\nocite{LiL2023BayesianForecast}
\nocite{WangZ2022EscalatorAccident}
\nocite{WangX2023DistributedARIMA}
\nocite{AndererM2022HierarchicalForecasting}
\nocite{JanewayMG2021ClinicalDiagnostic}
\nocite{PetropoulosF2022ForecastingTheory}
\nocite{KangY2022ForecastForecasts}
\nocite{TalagalaTS2022FFORMPPFeaturebased}
\nocite{ZhuX2021LeastSquareApproximation}
\nocite{WangX2022UncertaintyEstimation}
\nocite{KangY2021DejaVu}
\nocite{HaoC2020BilinearReduced}
\nocite{LiX2020ForecastingTime}
\nocite{KangY2020GRATISGeneRAting}
\nocite{kang2020statcompcn}
\nocite{kang2020fppcn}
\nocite{KalesanB2020IntersectionsFirearm}
\nocite{BaileyHM2019ChangesPatterns}
\nocite{LiF2019CreditRisk}
\nocite{LiF2018ImprovingForecasting}
\nocite{PinoEC2018CohortProfile}
\nocite{li2016distributedcn}
\nocite{LiF2013BayesianModeling}
\nocite{LiF2013EfficientBayesian}
\nocite{LiF2011ModellingConditional}
\nocite{LiF2010FlexibleModeling}

\printbibliography[heading=none]

\newpage
\section{计算机软件}
李丰老师和他的团队开发具有自主产权、适用于海量数据的开源统计和机器学习软件。详情请访问软件仓库 \url{https://github.com/feng-li}。

\begin{center}
\resizebox{\textwidth}{!}{
\begin{tabular}{lp{6cm}cclp{3cm}}
  \toprule
  Package     & Description                                                                                                                             & Language & Environment & Available On & Related Publication                                         \\
  \midrule
gratis        & Efficient algorithms for generating time series with diverse and controllable characteristics (Selected in R Task View for Time Series) & R        & All         & CRAN, GitHub & \citet{KangY2020GRATISGeneRAting}
                                                                                                                                                                                                                                                              \\
videofeatures & Efficient algorithms for generating time series features from video and voice data                                                      & Python   & GPU         & GitHub                                                                     \\
febama        & Feature-based Bayesian Forecasting Model Averaging                                                                                      & R        & All         & GitHub       & \citet{LiL2023BayesianForecast}                             \\
fide          & Feature-based Intermittent DEmand forecasting                                                                                           & R        & All         & GitHub       & \citet{LiL2023FeaturebasedIntermittent}                     \\
fuma          & Forecast uncertainty based on model averaging                                                                                           & R        & All         & GitHub       & \citet{WangX2022UncertaintyEstimation}                      \\
fformpp       & Feature-based FORecast Model Performance Prediction                                                                                     & R        & All         & GitHub       & \citet{TalagalaTS2022FFORMPPFeaturebased}                   \\
dng           & Distribution and Gradients for Skewed Distributions (Selected in R Task View for Probability Distributions)                             & R        & All         & CRAN, GitHub & \citet{LiF2010FlexibleModeling,LiF2011ModellingConditional} \\
pyhts         & A python package for hierarchical forecasting                                                                                           & Python   & All         & GitHub, PyPi & \citet{ZhangB2023OptimalReconciliation}                     \\
dlsa          & Distributed Least Squares Approximation implemented with Apache Spark                                                                   & Python   & Spark       & GitHub       & \citet{ZhuX2021LeastSquareApproximation}                    \\
darima        & Distributed ARIMA models implemented with Apache Spark                                                                                  & Python   & Spark       & GitHub       & \citet{WangX2023DistributedARIMA}                           \\
dts           & Distributed time series models implemented with distributed FFTs                                                                        & Python   & Spark       & GitHub                                                                     \\
dqr           & Distributed Quantile Regression by Pilot Sampling and One-Step Updating                                                                 & Python   & Spark       & GitHub       & \citet{PanR2022NoteDistributed}                             \\
cdcopula      & Covariate-dependent copula models                                                                                                       & R        & All         & GitHub       & \citet{LiF2018ImprovingForecasting}                         \\
movingknots   & Efficient Bayesian Multivariate Surface Regression                                                                                      & R        & All         & GitHub       & \citet{LiF2013EfficientBayesian}                            \\
flutils       & A collection of R functions which is required from my other packages                                                                    & R        & All         & GitHub                                                                     \\
GSM           & Flexible Modeling of Conditional Distributions using Smooth Mixtures of Asymmetric Student T Densities                                  & Matlab   & All         & GitHub       & \citet{LiF2010FlexibleModeling}                             \\
  \bottomrule
\end{tabular}
}
\end{center}

\end{refsection}

\newpage
\section{讲授课程}
(\emph{标*为全英文授课。获取课程讲义请访问} \url{http://feng.li/teaching/})

\begin{center}
  \resizebox{\textwidth}{!}{
    \begin{tabular}{llll}
      \toprule
      \textbf{课程名称}                         & \textbf{对象}         & \textbf{开课单位}              & \textbf{时间}    \\
      \midrule
      \textbf{AI驱动的预测}                     & 本科、硕士、博士、MBA & 北京大学光华管理学院           & 2025 --          \\
      \textbf{大数据计算与预测}                 & 硕士、博士            & 北京大学光华管理学院           & 2025 --          \\
      \textbf{分布式存储与计算}                 & 硕士、博士            & 北京大学光华管理学院           & 2020 -- 2024     \\
                                                                                                                            \\
      \textbf{统计计算}                         & 本科生                & 中央财经大学                   & 2014 -- 2024     \\
      \multicolumn{4}{l}{\footnotesize(中央财经大学精品实验课、课程思政专项立项课程)}                                     \\
      \textbf{数据科学工具}                     & 本科生                & 中央财经大学                   & 2018 -- 2023     \\
      \footnotesize{(中央财经大学核心通识课)} &                       &                                &                  \\
      \textbf{Python程序设计与数据挖掘}         & MBA 研究生            & 中央财经大学商学院             & 2022 -- 2024     \\
                                                & 研究生                & 中央财经大学粤港澳大湾区研究院 & 2021 -- 2022     \\
      \textbf{大数据分布式计算}                 & 本科生/研究生         & 中央财经大学                   & 2020秋 -- 2024   \\
                                                & 研究生                & 首都高校大数据联合培养硕士     & 2014秋 -- 2019秋 \\
                                                & 研究生                & 首都经济贸易大学               & 2020春 --2021春  \\
      \textbf{大数据计算机基础}                 & 研究生                & 首都高校大数据联合培养硕士     & 2015秋           \\
      \textbf{应用统计案例选讲}                 & 研究生                & 中央财经大学                   & 2016-2018春      \\
      \textbf{贝叶斯分析}$^*$                   & 研究生                & 斯德哥尔摩大学                 & 2013春           \\
                                                & 本科生                & 中央财经大学                   & 2017春           \\
                                                & 研究生                & 首都师范大学                   & 2017春           \\
      \textbf{时间序列}                         & 研究生                & 中央财经大学                   & 2015 -- 2016春   \\
      \textbf{计量经济学}$^*$                   & 本科生                & 中央财经大学                   & 2013秋 -- 2015秋 \\
      \textbf{现代统计软件}                     & 本科生                & 中央财经大学                   & 2014春           \\
      \textbf{专业英语}(现代统计前沿)$^*$     & 研究生/博士生         & 中央财经大学                   & 2013 -- 2016秋   \\
      \textbf{R语言编程}$^*$                    & 本科生/研究生         & 瑞典林雪平大学                 & 2012 春          \\
      \textbf{回归分析}$^*$                     & 本科生                & 瑞典斯德哥尔摩大学             & 2008--2013       \\
      \textbf{时间序列}$^*$                     & 本科生                & 瑞典斯德哥尔摩大学             & 2008--2013       \\
      \bottomrule
    \end{tabular}
  }
\end{center}

\section{主持教改课题}

\begin{itemize}
\item 2023年~~中央财经大学 教学方法研究项目:AI时代下的《统计计算》课程思政建设。
\item 2019年~~中央财经大学 数据科学大数据技术方向新开课项目。
\item 2018年~~中央财经大学 精品实验课 《统计计算》项目(结项优秀)。
\end{itemize}

\newpage
\section{主办会议}

\begin{itemize}

\item The 2025 International Symposium on Forecasting, 2025年6月29日,中国北京。
\item The 2024 PKU workshop on Modern Bayesian Computation, 2024年12月9日,中国北京。

\item  The 2017 Beijing Workshop on Forecasting, 2017年11月18日,中国北京。
\item 中国数量经济学会2016年会,2016年10月15-17日,中国北京。

\item International Symposium on Financial Engineering and Risk Management 2014,
  2014年6月27-28日, 中国北京。

\item The Swedish Research Students Conference in Statistics, 2013, 瑞典斯德哥尔摩。
\end{itemize}

\section{部分会议发言}

\begin{etaremune}[itemsep=0ex,parsep=0pt]

\item Quarterly Forecasting Forum (International Institute of Forecasters), University of Bath School of Management, February 14, 2025, UK.

\item The 7th International Conference on Econometrics and Statistics (EcoSta 2024), July 17, 2014, Beijing, China.

\item The 44th International Symposium on Forecasting, June 30-July 02, 2024, Dijon, France.

\item The 23rd IEEE International Conference on Data Mining, December 1 - 4, 2023, Shanghai, China.

\item The 9th International Forum on Statistics (RUC IFS 2023), July 14-15, 2023, Beijing, China.

\item The 2023 ICSA China Conference, June 30 – July 3, 2023, Sichuan, China.

\item The 2021 World Meeting of the International Society for Bayesian Analysis, Jun
  28— Jul 02, 2021。邀请发言人。

\item The 41st International Symposium on Forecasting, 2021年7月27-30。邀请发言人。

\item The 40th International Symposium on Forecasting,  2020年10月25-11月5日。邀请发言人。

\item Twelfth International Conference on Monte Carlo Methods and Application (MCM 2019),
  澳大利亚悉尼, 2019年7月8-12日。邀请发言人。

\item The 39th International Symposium on Forecasting, Thessaloniki, 希腊 2019年6月16-19。邀请发言人。

\item ICSA Conference on Data Science, January 11-13, 2019, Xishuangbanna, China。邀请发言人。

\item International Symposium on Financial Engineering and Risk Management 2018, 2018年6月13,
  2018, 中国上海。邀请发言人。

\item School of Data Science, Fudan University, Oct 28-30, 2017, 中国上海。邀请发言人。

\item IMS-China International Conference on Statistics and Probability, June 28 – July 1,
  2017, 中国南宁。邀请发言人。


\item The 2016 World Meeting of the International Society for Bayesian Analysis, Jun
  13—17, 2016, Sardinia, 意大利。邀请发言人。

\item IMS-China International Conference on Statistics and Probability, June 1-4, 2015,
  Kunming, China. 邀请发言人。

\item 第十届全国概率统计会议,2014年10月17日-21日,中国山东。邀请发言人。

\item International Symposium on Financial Engineering and Risk Management 2014, 2014年6月27,
  2014, 中国北京。邀请发言人。

\item 大数据决策与合规论坛,2014年5月20-21日,中国深圳。邀请发言人。

\item Guanghua School of Management Peking University, Oct 14, 2013, 中国北京。邀请发言人。

\item The 2012 World Meeting of the International Society for Bayesian
  Analysis, 2012年6月25日---29日, 日本。

\item The third Linnaeus University Workshop in Stochastic Analysis and
  Applications, 2012年5月24日---25日, 瑞典。邀请发言人。

\item Seminar at Department of Energy and Technology, Swedish
  University of Agricultural Sciences, 2012年4月16, 瑞典。

\item Workshop on ``Analysis of High-Dimensional Data'',
  Jönköping International Business School, 2012年2月16日---17日,瑞典。
  邀请发言人。

\item The LiU Seminar Series in Statistics and Mathematical Statistics,
  Linköping University, 2011年10月11日, 瑞典。 邀请发言人。

\item The 42nd Winter Conference in Statistics --- Incomplete data:
  semi-parametric and Bayesian methods, 2011年3月6日---10日,
  瑞典。 邀请发言人。

\item The 2010 World Meeting of the International Society for Bayesian
  Analysis, 2010年6月3日---8日, 西班牙。

\item Seminar at Department of statistics, Uppsala University, 2009年9
  月16日, 瑞典。

\end{etaremune}


\section{学术访问}
%\emph{(仅列出部分近期活动})

\begin{itemize}
\item 访问美国罗格斯大学,\emph{2025年10月}。
\item 访问英国巴斯大学管理学院,\emph{2025年2月}。
\item 访问澳大利亚悉尼大学商学院,\emph{2024年10月}。
\item 参加美国经济学年会,\emph{2016年1月}。
\item 访问加拿大多伦多大学,\emph{2014年8月}。
\item 访问瑞典斯德哥尔摩大学,\emph{2013年10月}。

\item 访问瑞典林雪平大学,\emph{2011年9月1日 --- 2012年 2月29日}。

\item 参与短期博士生课程: ``Introduction to Bayesian Analysis and
  MCMC, and,Hierarchical Modelling of Spatial and Temporal Data'',讲
  师: Alan Gelfand (杜克大学) Sujit Sahu (南安普顿大学), June
  7---10,2011, 英国。

\item 参与短期博士生课程: ``Semi-Parametric Bayesian Inference in
  Econometrics'' 讲师: Peter Rossi (芝加哥大
  学), 2009年5月 27---29日,2009,荷兰。

\item 瑞典中央银行会议 ``Modeling and Forecasting Economic and
  Financial Time Series with State Space
  models'', 2008年10月 17---18日,瑞典斯德哥尔摩。
\end{itemize}

\section{获奖情况}
\begin{itemize}

\item Grand Prize for the Point Forecasting Track in the Tourism Forecasting Competition II, 2025年3月。
\item 第二届全国高校经管类实验教学案例大赛二等奖,2017年12月。
\item The 2014 Cramér Prize,瑞典统计学会,2014年3月。

\item 国际贝叶斯协会青年旅行奖励,2012年6月。

\item 瑞典 Knut and Alice Wallenberg 基金会奖励, 2011年8月。

\item 北京市级优秀毕业生, 2007年7月。
\end{itemize}

\section{学术评审}

\begin{itemize}
\item   担任国内外知名高校博士论文答辩评委,为北京大学、复旦大学、澳大利亚蒙纳士大学等10余篇博士论文撰写独立评审意见。
\item   担任联合国大数据黑客松大赛指导老师(2022年-2023年)
\item   担任统计学、计算机与人工智能、管理学等领域50余种国际期刊的匿名评审,过去五年撰写150多个高质量匿名论文评审意见。期刊评审记录详见 \url{https://orcid.org/0000-0002-4248-9778}。以下列出部分常审稿期刊列表。
  \begin{itemize}
  \item Journal of the American Statistical Association
  \item Journal of the Royal Statistical Society: Series B (Statistical Methodology)
  \item International Journal of Forecasting
  \item Bayesian Analysis
  \item Journal of Business and Economics Statistics
  \item Annals of Applied Statistics
  \item Computational Statistics and Data Analysis
  \item IEEE Transactions on Pattern Analysis and Machine Intelligence (TPAMI)
  \item Neurocomputing
  \item Pattern Recognition
  \item Omega - The International Journal of Management Science
  \end{itemize}
\end{itemize}

\section{其他学术兼职}
\begin{tabular}{l p{0.8\textwidth} l}
  2023--     & 北京市统计学会理事                                             \\
  2023--     & 中国土木工程学会风险管理分会理事                               \\
  2018--2023 & 全国工业统计学教学研究会常务理事、中国青年统计学家协会副秘书长 \\
  2014--2023 & 中国统计教育学会高等教育分会副秘书长。                         \\
  2013--2014 & 2014年金融工程与风险管理国际研讨会执行秘书。                   \\
\end{tabular}

% \section{联系人}
% \begin{itemize}
% \item Mattias Villani教授, 瑞典斯德哥尔摩大学统计学系。电子邮件:
%   \url{mattias.villani@stat.su.se}。
% \item Fan Yang Wallentin 教授, 瑞典乌普萨拉大学统计系。电子邮件:
%   \url{fan.yang@statistik.uu.se}。
% \item Robert Kohn教授,澳洲社会科学院院士, 澳大利亚新南威尔士大学澳大利亚商学
%   院。电子邮件:\url{r.kohn@unsw.edu.au}。
% \end{itemize}

% \footnotesize{\emph{* 联系人联系方式可以单独索取。}}
\end{document}
