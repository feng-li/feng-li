\documentclass[twoside,a4paper,10pt]{amsart}

%%%%%%%%%%%%%%%%%% Non-Latin Languages Support %%%%%%%%%%%%%%%%%%%%%%%%%%%%%%%%
\usepackage[fontset=adobe]{ctex}
% \usepackage{ctex}
%%%%%%%%%%%%%%%%%%%%%%%%%%%%%%%%%%%%%%%%%%%%%%%%%%%%%%%%%%%%%%%%%%%%%%%%%%%%%%%

\usepackage[twoside]{geometry}
\geometry{lmargin=0.5in,rmargin=0.5in,tmargin=1in,bmargin=1in}
\usepackage[raggedright]{titlesec}

%% BibLaTeX
\usepackage[backend=biber, style=authoryear, sorting=ydnt,
dashed=none, maxcitenames=2, maxbibnames=99, giveninits=true, hyperref=true,
natbib=true,uniquelist=false, uniquename=false]{biblatex}
\addbibresource{full.bib}
\addbibresource{references-feng.bib}


\usepackage{titlesec}
\titlespacing{\section}{0pt}{0pt}{0pt}

\usepackage{bm}
\usepackage{amsmath}
\usepackage{amsthm}
\usepackage{amstext}
\usepackage{fontawesome}
\usepackage{graphicx}
\usepackage[colorlinks=true]{hyperref}
\hypersetup
{
  pdfauthor={Feng Li, 李丰},
  pdfsubject={Feng Li, Curriculum Vitae; 李丰 个人简历},
  pdftitle={Feng Li, Curriculum Vitae; 李丰 个人简历},
  pdfkeywords={Feng Li, 李丰, Stockholm University, Statistics, CV, Resume, http://feng.li/}
}

%% Fonts
% \usepackage[garamond]{mathdesign}
% \DeclareRobustCommand\nobreakspace{\leavevmode\nobreak\ } % Bug in fontspec
% \usepackage{inconsolata}


%% Fancy symbols
\usepackage{marvosym}

%% Fancy header and footer
\usepackage{fancyhdr}
\pagestyle{fancy}

\fancyhead[RO,LE]{李丰}
\fancyhead[C]{}
\fancyhead[LO,RE]{\emph{个人简历}}

\fancyfoot[C]{}
\fancyfoot[LO,RE]{}
\fancyfoot[RO,LE]{\vspace{0.5cm}\thepage}

\fancypagestyle{plain}{%
  \fancyhf{} % clear all header and footer fields
  \fancyfoot[LO,RE]{\vspace{0.5cm}\emph{修改日期: \today \\最新个人简历请访问} \url{http://feng.li/}}
  \fancyfoot[C]{}
  \fancyfoot[LE,RO]{\vspace{0.5cm}\thepage}
  \renewcommand{\headrulewidth}{0pt}
  \renewcommand{\footrulewidth}{0pt}}

\begin{document}
\thispagestyle{plain}
\section*{\Huge{李丰~博士}}
\begin{center}
  \emph{\huge 个人简历}
\end{center}
\rule{\textwidth}{.01cm}

\section*{个人信息}
\begin{tabular}{l p{4cm} l  l l}
  姓名:& 李丰& 外文论文署名: &   Feng Li\\
  % 性别: & 男 & 出生日期: &1984年1月16日\\
  性别: & 男  &掌握语言: & 中文(母语),英语(流利)\\
\end{tabular}

\section*{工作单位}

\begin{itemize}
\item [] 中央财经大学统计与数学学院,副院长,副教授,硕士研究生导师
\end{itemize}

\section*{联系方式}

\texttt{
  \begin{tabular}{ l l |  l  l l l}
    北京市昌平区沙河高教园   &  & \Email   & \href{mailto:feng.li@cufe.edu.cn}{feng.li@cufe.edu.cn} \\
    中央财经大学(沙河校区) &  & \faGlobe & \url{http://feng.li/}                                  \\
    统计与数学学院           &  & \Telefon & +86-(0)10-6117-6189                                    \\
    邮编:102206             &  &          &
  \end{tabular}}

\section*{教育背景}

\begin{tabular}{ l  p{0.75\textwidth} l}
  2008 --- 2013 &\textbf{统计学~博士},瑞典斯德哥尔摩大学统计学系\\
  & 博士论文:\emph{贝叶斯条件密度估计方法}\\
  & 导师: Mattias Villani教授\\
  & 答辩对手:Sylvia Frühwirth-Schnatter 教授\\

  2007 --- 2008 & \textbf{统计学~硕士}, 瑞典达拉那大学统计学系\\

  2003 --- 2007 & \textbf{统计学~本科}, 中国人民大学统计学院\\
\end{tabular}

\section*{研究兴趣}

\begin{itemize}
\item 大数据分布式学习
\item 贝叶斯推断与统计计算
\item 计量经济与预测方法

\end{itemize}

\section*{发表论文}
\nocite{wang2021uncertainty}
\nocite{kang2021}
\nocite{hao2020bilinear}
\nocite{kang2020gratis}
\nocite{bailey2019changes}
\nocite{li2019credit}
\nocite{li2018improving}
\nocite{pino2018cohort}
\nocite{li2016distributed}
\nocite{li2013bayesian}
\nocite{li2013efficient}
\nocite{li2011modeling}
\nocite{li2010flexible}
\printbibliography[heading=none,  nottype=software]

\section*{软件}
\nocite{gratis}
\nocite{spark-dlsa}
\printbibliography[heading=none, type=software]
%{\footnotesize (软件更新 \url{https://github.com/feng-li})}

\section*{科研课题}

\begin{itemize}

\item 2020年国家自然科学基金面上项目:\emph{中医药临床疗效评价中基于目标值法的单臂临床研究方法体
    系的构建},主要参与人,在研。

\item 国家自然科学基金青年项目:\emph{贝叶斯柔性密度方法及其在高维金融数据中的应用},项目负责人,
  结项。

\item 2014年教育部人文社科研究项目:\emph{贝叶斯弹性高维密度方法在复杂数据的研
    究},项目负责人,结项。

\item 2014年国家自然科学基金青年项目: \emph{复发事件的均值模型和纵向数据的分位数回归的统计
    与推断},主要参加人,结项。

\item 2014年国家自然科学基金青年项目: \emph{公司财务困境预警模型研究:基于财务波动信息的区
    间数据刻画方法},主要参加人,结项。

\item 2014年国家自然科学基金面上项目:\emph{货币总量转向信用总量:全球虚拟经济与实体经济背
    离机理与宏观政策应对},主要参加人,结项。

\end{itemize}

\section*{所授课程}
(\emph{标*为全英文授课。获取课程讲义请访问} \url{http://feng.li/teaching/})

\begin{tabular}{l l ll}
  \textbf{统计计算} & 本科生 & 中央财经大学 & 2014--2018春\\
  \textbf{大数据分布式计算} & 研究生 & 专硕大数据联合培养硕士 & 2014秋--2017秋\\
  \textbf{大数据计算机基础} & 研究生 & 专硕大数据联合培养硕士 & 2015秋\\
  \textbf{应用统计案例选讲}&研究生& 中央财经大学& 2016-2018春\\
  \textbf{贝叶斯分析}$^*$&研究生& 斯德哥尔摩大学&2013春\\
                            &本科生&中央财经大学&2017春\\
                            &研究生&首都师范大学&2017春\\
  \textbf{时间序列}& 研究生 & 中央财经大学 & 2015--2016春\\
  \textbf{计量经济学}$^*$ &本科生&  中央财经大学& 2013秋--2015秋\\
  \textbf{现代统计软件}&本科生& 中央财经大学& 2014春\\
  \textbf{专业英语}(现代统计前沿)$^*$&研究生/博士生& 中央财经大学& 2013-2016秋\\

  \textbf{R语言编程}$^*$&本科生/研究生&林雪平大学&  2012 春\\
  \textbf{回归分析}$^*$&本科生&斯德哥尔摩大学&2008--2013\\
  \textbf{时间序列}$^*$&本科生&斯德哥尔摩大学&2008--2013\\


\end{tabular}

\section*{主办会议}

\begin{itemize}
\item  The 2017 Beijing Workshop on Forecasting, 2017年11月18日,中国北京。
\item 中国数量经济学会2016年会,2016年10月15-17日,中国北京。

\item International Symposium on Financial Engineering and Risk Management 2014,
  2014年6月27-28日, 中国北京。

\item The Swedish Research Students Conference in Statistics, 2013, 瑞典斯德哥尔摩。

\end{itemize}

\section*{会议发言}

\emph{(仅列出部分会议发言)}

\begin{itemize}

\item International Symposium on Financial Engineering and Risk Management 2018, 2018年6月13,
  2018, 中国上海。邀请发言人。


\item School of Data Science, Fudan University, Oct 28-30, 2017, 中国上海。

\item IMS-China International Conference on Statistics and Probability, June 28 – July 1,
  2017, 中国南宁。


\item The 2016 World Meeting of the International Society for Bayesian Analysis, Jun
  13—17, 2016, Sardinia, 意大利。

\item IMS-China International Conference on Statistics and Probability, June 1-4, 2015,
  Kunming, China. 邀请发言人。

\item 第十届全国概率统计会议,2014年10月17日-21日,中国山东。邀请发言人。

\item International Symposium on Financial Engineering and Risk Management 2014, 2014年6月27,
  2014, 中国北京。邀请发言人。

\item 大数据决策与合规论坛,2014年5月20-21日,中国深圳。邀请发言人。

\item Guanghua School of Management Peking University, Oct 14, 2013, 中国北京。

\item The 2012 World Meeting of the International Society for Bayesian
  Analysis, 2012年6月25日---29日, 日本。

\item The third Linnaeus University Workshop in Stochastic Analysis and
  Applications, 2012年5月24日---25日, 瑞典。邀请发言人。

\item Seminar at Department of Energy and Technology, Swedish
  University of Agricultural Sciences, 2012年4月16, 瑞典。

\item Workshop on ``Analysis of High-Dimensional Data'',
  Jönköping International Business School, 2012年2月16日---17日,瑞典。
  邀请发言人。

\item The LiU Seminar Series in Statistics and Mathematical Statistics,
  Linköping University, 2011年10月11日, 瑞典。 邀请发言人。

\item The 42nd Winter Conference in Statistics --- Incomplete data:
  semi-parametric and Bayesian methods, 2011年3月6日---10日,
  瑞典。 邀请发言人。

\item The 2010 World Meeting of the International Society for Bayesian
  Analysis, 2010年6月3日---8日, 西班牙。

\item Seminar at Department of statistics, Uppsala University, 2009年9
  月16日, 瑞典。

\end{itemize}


\section*{学术活动}
\emph{(仅列出部分近期活动})

\begin{itemize}
\item 参加美国经济学年会,\emph{2016年1月}。
\item 访问多伦多大学,\emph{2014年8月}。
\item 访问斯德哥尔摩大学,\emph{2013年10月}。

\item 访问瑞典林雪平大学,\emph{2011年9月1日 --- 2012年 2月29日}。

\item 参与短期博士生课程: ``Introduction to Bayesian Analysis and
  MCMC, and,Hierarchical Modelling of Spatial and Temporal Data'',讲
  师: Alan Gelfand (杜克大学) Sujit Sahu (南安普顿大学), June
  7---10,2011, 英国。

\item 参与短期博士生课程: ``Semi-Parametric Bayesian Inference in
  Econometrics'' 讲师: Peter Rossi (芝加哥大
  学), 2009年5月 27---29日,2009,荷兰。

\item 瑞典中央银行会议 ``Modeling and Forecasting Economic and
  Financial Time Series with State Space
  models'', 2008年10月 17---18日。
\end{itemize}

\section*{计算机水平}
\begin{itemize}

\item 掌握基于Hadoop/Spark的大数据计算平台及其统计方法开发。

\item  精通 \textbf{R},Matlab熟练使用SAS,Julia和Python。

\item  掌握GNU/Linux服务器管理,有大型 Linux 集群使用经验。

\item  熟练使用C/C++。
\end{itemize}

\section*{获奖情况}
\begin{itemize}

\item 第二届全国高校经管类实验教学案例大赛二等奖,2017年12月。
\item The 2014 Cramér Prize,瑞典统计学会,2014年3月。

\item 国际贝叶斯协会青年旅行奖励,2012年6月。

\item 瑞典 Knut and Alice Wallenberg 基金会奖励, 2011年8月。

\item 北京市级优秀毕业生, 2007年7月。
\end{itemize}

\section*{担任期刊审稿}
\begin{itemize}
\item  \emph{Journal of Business and Economics Statistics}
\item  \emph{International Journal of Forecasting}
\item  \emph{Journal of Statistical Computation and Simulation}
\item  \emph{Australian \& New Zealand Journal of Statistics}
\item  \emph{Journal of Official Statistics}
\item  \emph{Quantitative Finance}
\item  \emph{Information Sciences}
\item  \emph{Studies in Nonlinear Dynamics \& Econometrics}
\end{itemize}

\section*{其他学术职责}
\begin{tabular}{l p{0.8\textwidth} l}


  2015-- & 北京大数据协会理事。\\

  2014-- & 中国统计教育学会高等教育分会副秘书长。\\

  2013--2014 & 2014年金融工程与风险管理国际研讨会执行秘书。\\

  %% 2012--2013 & 瑞典首届统计学博士会议发起和组织者。
  %% \url{http://gauss.stat.su.se/doktorand/srss2013/}.\\

  %% 2012---2013 & 瑞典斯德哥尔摩大学统计学博士学术讲座发起和组织者(常规
  %% 学术活动,每两周一次)。\url{http://gauss.stat.su.se/doktorand/seminars/}.\\

  %% 2012---2013 & 瑞典斯德哥尔摩大学统计系IT改革小组成员。\\

\end{tabular}

% \section*{联系人}
% \begin{itemize}
% \item Mattias Villani教授, 瑞典林雪平大学计算机系。电子邮件:
%   \url{mattias.villani@liu.se}。
% \item Fan Yang Wallentin 教授, 瑞典乌普萨拉大学统计系。电子邮件:
%   \url{fan.yang@statistik.uu.se}。
% \item Robert Kohn教授,澳洲社会科学院院士, 澳大利亚新南威尔士大学澳大利亚商学
%   院。电子邮件:\url{r.kohn@unsw.edu.au}。
% \end{itemize}

% \footnotesize{\emph{* 联系人联系方式可以单独索取。}}
\end{document}

%%% Local Variables:
%%% TeX-engine: xetex
%%% End:
