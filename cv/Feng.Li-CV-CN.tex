\documentclass[twoside,a4paper,10.5pt]{article}

%% Fonts
\usepackage[fontset=adobe]{ctex}
\setmainfont{Garamond Premier Pro}
% \setmainfont{Adobe Garamond Pro}
\usepackage{fontawesome}

%% Geometry
\usepackage[twoside]{geometry}
\geometry{lmargin=0.5in,rmargin=0.5in,tmargin=1in,bmargin=1in}
\usepackage{setspace}
\setstretch{1.25}

%% BibLaTeX
\usepackage[backend=biber, style=authoryear, sorting=ydnt, url=false, dashed=none,
maxcitenames=2, maxbibnames=99, giveninits=false, hyperref=true,
natbib=true,uniquelist=false, uniquename=false]{biblatex}
\renewbibmacro{in:}{%
  \ifentrytype{article}{}{%
  \printtext{\bibstring{in}\intitlepunct}}}
\DeclareNameAlias{sortname}{given-family}

% replace and with &
\renewcommand*\finalnamedelim{\addspace\&\space}

% Number with brackets, bold italic for journal names
\DeclareFieldFormat[article]{number}{(#1)}
\DeclareFieldFormat{journaltitle}{\mkbibbold{\mkbibitalic{#1}}}
\DeclareFieldFormat{booktitle}{\mkbibbold{\mkbibitalic{#1}}}

% Reverse enumerated numbers to bibliography and keep the default sorting order.
\usepackage{etaremune}
\defbibenvironment{bibliography}{\begin{etaremune}[itemsep=1ex,parsep=0pt]} {\end{etaremune}} {\item}
\renewcommand{\labelenumi}{[\theenumi]}

\addbibresource{publications-feng.bib}


%% Highlight certain author name
\usepackage{xpatch}
\makeatletter
\newbibmacro*{name:bold}[2]{%
  \edef\blx@tmp@name{\expandonce#1, \expandonce#2}%
  \def\do##1{\ifdefstring{\blx@tmp@name}{##1}{\bfseries\listbreak}{}}%
  \dolistloop{\boldnames}}
\newcommand*{\boldnames}{}
\makeatother
\xpretobibmacro{name:family}{\begingroup\usebibmacro{name:bold}{#1}{#2}}{}{}
\xpretobibmacro{name:given-family}{\begingroup\usebibmacro{name:bold}{#1}{#2}}{}{}
\xpretobibmacro{name:family-given}{\begingroup\usebibmacro{name:bold}{#1}{#2}}{}{}
\xpretobibmacro{name:delim}{\begingroup\normalfont}{}{}
\xapptobibmacro{name:family}{\endgroup}{}{}
\xapptobibmacro{name:given-family}{\endgroup}{}{}
\xapptobibmacro{name:family-given}{\endgroup}{}{}
\xapptobibmacro{name:delim}{\endgroup}{}{}

% List your own name variants that are to be highlighted
\forcsvlist{\listadd\boldnames}
{{Li, Feng}, {Li, F.}}

%% Spacing
\usepackage{enumitem}
\setlist[enumerate,itemize]{nosep,listparindent=\parindent,leftmargin=2em}% Indented paragraphs within lists

\usepackage[compact,raggedright]{titlesec}
\titlespacing{\section}{0pt}{0ex}{0ex}
\titlespacing{\subsection}{0pt}{0ex}{0ex}
\titlespacing{\subsubsection}{0pt}{0ex}{0ex}
\setlength{\parskip}{.0em}
\setlength{\parindent}{1em}

\usepackage{booktabs}
\usepackage{graphicx}
\usepackage[colorlinks=true]{hyperref}
\hypersetup
{
  pdfauthor={Feng Li, 李丰},
  pdfsubject={Feng Li, Curriculum Vitae; 李丰 个人简历},
  pdftitle={Feng Li, Curriculum Vitae; 李丰 个人简历},
  pdfkeywords={Feng Li, 李丰, Stockholm University, Statistics, CV, Resume, http://feng.li/}
}

%% Fancy header and footer
\usepackage{fancyhdr}
\pagestyle{fancy}

\fancyhead[RO,LE]{李丰}
\fancyhead[C]{}
\fancyhead[LO,RE]{\emph{个人简历}}

\fancyfoot[C]{}
\fancyfoot[LO,RE]{}
\fancyfoot[RO,LE]{\vspace{0.5cm}\thepage}

\fancypagestyle{plain}{%
  \fancyhf{} % clear all header and footer fields
  \fancyfoot[LO,RE]{\vspace{0.5cm}\tiny\emph{\today 更新。最新个人简历请访问}  \url{https://feng.li/}}
  \fancyfoot[C]{}
  \fancyfoot[LE,RO]{\vspace{0.5cm}\thepage}
  \renewcommand{\headrulewidth}{0pt}
  \renewcommand{\footrulewidth}{0pt}}

\begin{document}
\thispagestyle{plain}
\section*{\Huge{李丰~~博士}}
\begin{center}
  \emph{\huge 个人简历}
\end{center}
\rule{\textwidth}{.01cm}

\section*{个人信息}
\begin{tabular}{l p{4cm} l  l l}
  姓名:   & 李丰 & 外文论文署名: & Feng Li                                 \\
  % 性别: & 男   & 出生日期:     & 1984年1月16日                           \\
  性别:   & 男   & 掌握语言:     & 中文(母语),英语(流利)              \\
  \includegraphics[height=1em]{orcid-logo}   & \multicolumn{3}{l}{\href{https://orcid.org/0000-0002-4248-9778}{0000-0002-4248-9778}} \\
\end{tabular}

\section*{工作单位}

\begin{itemize}
\item [] 中央财经大学统计与数学学院,副院长,副教授,硕士研究生导师
\end{itemize}


\section*{联系方式}

\begin{tabular}{ l l |  l  l l l}
  北京市昌平区沙河高教园   &  & \faEnvelopeO & \href{mailto:feng.li@cufe.edu.cn}{\url{feng.li@cufe.edu.cn}} \\
  中央财经大学(沙河校区) &  & \faHome & \url{https://feng.li}                                        \\
  统计与数学学院           &  & \faGroup & \url{https://kllab.org}                                      \\
  邮编:102206             &  & \faPhone & \texttt{+86-(0)10-6117-6189}                                 \\
\end{tabular}

\section*{教育背景}

\begin{tabular}{ l  p{0.75\textwidth}}
  2008 --- 2013 & \textbf{统计学~博士},瑞典斯德哥尔摩大学统计学系                                       \\
                & 博士论文:\emph{贝叶斯条件密度建模}(\emph{Bayesian Modeling of Conditional Densities}) \\
                & (获评2014年瑞典最佳统计学博士论文——Cramér Prize)                                       \\
                & 导  师: Mattias Villani教授                                                         \\
                & 答辩主席:奥地利维也纳经济大学Sylvia Frühwirth-Schnatter 教授                          \\
  2007 --- 2008 & \textbf{统计学~硕士}, 瑞典达拉那大学统计学系                                          \\
  2003 --- 2007 & \textbf{统计学~本科}, 中国人民大学统计学院                                            \\
\end{tabular}

\section*{研究兴趣}

大数据分布式学习 $\circ$ 贝叶斯推断与统计计算 $\circ$ 计量经济学与预测方法

\section*{科研课题}

\begin{itemize}

\item 2020年国家自然科学基金面上项目:\emph{中医药临床疗效评价中基于目标值法的单臂临床研究方法体
    系的构建},主要参与人,在研。

\item 2015年国家自然科学基金青年项目:\emph{贝叶斯柔性密度方法及其在高维金融数据中的应用},项目负责人,
  结项。

\item 2014年教育部人文社科研究项目:\emph{贝叶斯弹性高维密度方法在复杂数据的研
    究},项目负责人,结项。

\item 2014年国家自然科学基金青年项目: \emph{复发事件的均值模型和纵向数据的分位数回归的统计与推断},主要参加人,结项。

\item 2014年国家自然科学基金青年项目: \emph{公司财务困境预警模型研究:基于财务波动信息的区间数据刻画方法},主要参加人,结项。

\item 2014年国家自然科学基金面上项目:\emph{货币总量转向信用总量:全球虚拟经济与实体经济背
    离机理与宏观政策应对},主要参加人,结项。

\end{itemize}


\section*{学术发表}
\begin{refsection}

\nocite{li2022bayesian_ijf}
\nocite{wang2022escalator_safety}
\nocite{wang2022distributed_ijf}
\nocite{anderer2022forecasting_ijf}
\nocite{janeway2021clinical_jad}
\nocite{petropoulos2021forecasting_ijf}
\nocite{kang2022forecast_ejor}
\nocite{pan2021note_jbes}
\nocite{talagala2022fformpp_ijf}
\nocite{zhu2021least_jcgs}
\nocite{wang2022uncertainty_jors}
\nocite{kang2021deja_jbr}
\nocite{hao2020bilinear_ced}
\nocite{li2020forecasting_eswa}
\nocite{kang2020gratis_sam}
\nocite{kang2020statscompcn}
\nocite{li2020fppcn}
\nocite{kalesan2020intersections_jsr}
\nocite{bailey2019changes_plosone}
\nocite{li2019credit_cef}
\nocite{li2018improving_ijf}
\nocite{pino2018cohort_bmj}
\nocite{li2016distributedcn}
\nocite{li2013bayesian}
\nocite{li2013efficient_sjs}
\nocite{li2011modeling_mixtures}
\nocite{li2010flexible_jspi}

\printbibliography[heading=none]
\end{refsection}

\section*{软件}
\begin{refsection}
\nocite{gratis}
\nocite{spark-dlsa}
\printbibliography[heading=none]
%{\footnotesize (软件更新 \url{https://github.com/feng-li})}
\end{refsection}

\section*{讲授课程}
(\emph{标*为全英文授课。获取课程讲义请访问} \url{http://feng.li/teaching/})

\begin{tabular}{l l ll}
  \toprule
  课程名称                                  & 对象          & 开课单位                       & 时间             \\
  \midrule
  \textbf{统计计算}                         & 本科生        & 中央财经大学                   & 2014 --          \\
  \footnotesize{(中央财经大学精品实验课)} &               &                                &                  \\
  \textbf{数据科学工具}                     & 本科生        & 中央财经大学                   & 2018 --          \\
  \footnotesize{(中央财经大学核心通识课)} &               &                                &                  \\
  \textbf{Python程序设计}                   & 研究生        & 中央财经大学粤港澳大湾区研究院 & 2021--           \\
  \textbf{大数据分布式计算}                 & 本科生/研究生 & 中央财经大学                   & 2020秋 --        \\
                                            & 研究生        & 北京大学光华管理学院           & 2019春 --        \\
                                            & 研究生        & 首都高校大数据联合培养硕士     & 2014秋 -- 2019秋 \\
                                            & 研究生        & 首都经济贸易大学               & 2020春 --        \\
  \textbf{大数据计算机基础}                 & 研究生        & 首都高校大数据联合培养硕士     & 2015秋           \\
  \textbf{应用统计案例选讲}                 & 研究生        & 中央财经大学                   & 2016-2018春      \\
  \textbf{贝叶斯分析}$^*$                   & 研究生        & 斯德哥尔摩大学                 & 2013春           \\
                                            & 本科生        & 中央财经大学                   & 2017春           \\
                                            & 研究生        & 首都师范大学                   & 2017春           \\
  \textbf{时间序列}                         & 研究生        & 中央财经大学                   & 2015 -- 2016春   \\
  \textbf{计量经济学}$^*$                   & 本科生        & 中央财经大学                   & 2013秋 -- 2015秋 \\
  \textbf{现代统计软件}                     & 本科生        & 中央财经大学                   & 2014春           \\
  \textbf{专业英语}(现代统计前沿)$^*$     & 研究生/博士生 & 中央财经大学                   & 2013 -- 2016秋   \\
  \textbf{R语言编程}$^*$                    & 本科生/研究生 & 瑞典林雪平大学                 & 2012 春          \\
  \textbf{回归分析}$^*$                     & 本科生        & 瑞典斯德哥尔摩大学             & 2008--2013       \\
  \textbf{时间序列}$^*$                     & 本科生        & 瑞典斯德哥尔摩大学             & 2008--2013       \\
\bottomrule
\end{tabular}

\section*{主办会议}

\begin{itemize}

  % \item The 2021 International Symposium on Forecasting -- China Hub, 2021年6月29日,中国
  %   北京。

\item  The 2017 Beijing Workshop on Forecasting, 2017年11月18日,中国北京。
\item 中国数量经济学会2016年会,2016年10月15-17日,中国北京。

\item International Symposium on Financial Engineering and Risk Management 2014,
  2014年6月27-28日, 中国北京。

\item The Swedish Research Students Conference in Statistics, 2013, 瑞典斯德哥尔摩。
\end{itemize}

\section*{会议发言}

\emph{(仅列出部分会议发言)}

\begin{itemize}

\item The 2021 World Meeting of the International Society for Bayesian Analysis, Jun
  28— Jul 02, 2021。邀请发言人。

\item The 41st International Symposium on Forecasting, 2021年7月27-30。邀请发言人。

\item The 40th International Symposium on Forecasting,  2020年10月25-11月5日。邀请发言人。

\item Twelfth International Conference on Monte Carlo Methods and Application (MCM 2019),
  澳大利亚悉尼, 2019年7月8-12日。邀请发言人。

\item The 39th International Symposium on Forecasting, Thessaloniki, 希腊 2019年6月16-19。邀请发言人。

\item ICSA Conference on Data Science, January 11-13, 2019, Xishuangbanna, China。邀请发言人。

\item International Symposium on Financial Engineering and Risk Management 2018, 2018年6月13,
  2018, 中国上海。邀请发言人。

\item School of Data Science, Fudan University, Oct 28-30, 2017, 中国上海。邀请发言人。

\item IMS-China International Conference on Statistics and Probability, June 28 – July 1,
  2017, 中国南宁。邀请发言人。


\item The 2016 World Meeting of the International Society for Bayesian Analysis, Jun
  13—17, 2016, Sardinia, 意大利。邀请发言人。

\item IMS-China International Conference on Statistics and Probability, June 1-4, 2015,
  Kunming, China. 邀请发言人。

\item 第十届全国概率统计会议,2014年10月17日-21日,中国山东。邀请发言人。

\item International Symposium on Financial Engineering and Risk Management 2014, 2014年6月27,
  2014, 中国北京。邀请发言人。

\item 大数据决策与合规论坛,2014年5月20-21日,中国深圳。邀请发言人。

\item Guanghua School of Management Peking University, Oct 14, 2013, 中国北京。邀请发言人。

\item The 2012 World Meeting of the International Society for Bayesian
  Analysis, 2012年6月25日---29日, 日本。

\item The third Linnaeus University Workshop in Stochastic Analysis and
  Applications, 2012年5月24日---25日, 瑞典。邀请发言人。

\item Seminar at Department of Energy and Technology, Swedish
  University of Agricultural Sciences, 2012年4月16, 瑞典。

\item Workshop on ``Analysis of High-Dimensional Data'',
  Jönköping International Business School, 2012年2月16日---17日,瑞典。
  邀请发言人。

\item The LiU Seminar Series in Statistics and Mathematical Statistics,
  Linköping University, 2011年10月11日, 瑞典。 邀请发言人。

\item The 42nd Winter Conference in Statistics --- Incomplete data:
  semi-parametric and Bayesian methods, 2011年3月6日---10日,
  瑞典。 邀请发言人。

\item The 2010 World Meeting of the International Society for Bayesian
  Analysis, 2010年6月3日---8日, 西班牙。

\item Seminar at Department of statistics, Uppsala University, 2009年9
  月16日, 瑞典。

\end{itemize}


\section*{学术访问}
%\emph{(仅列出部分近期活动})

\begin{itemize}
\item 参加美国经济学年会,\emph{2016年1月}。
\item 访问多伦多大学,\emph{2014年8月}。
\item 访问斯德哥尔摩大学,\emph{2013年10月}。

\item 访问瑞典林雪平大学,\emph{2011年9月1日 --- 2012年 2月29日}。

\item 参与短期博士生课程: ``Introduction to Bayesian Analysis and
  MCMC, and,Hierarchical Modelling of Spatial and Temporal Data'',讲
  师: Alan Gelfand (杜克大学) Sujit Sahu (南安普顿大学), June
  7---10,2011, 英国。

\item 参与短期博士生课程: ``Semi-Parametric Bayesian Inference in
  Econometrics'' 讲师: Peter Rossi (芝加哥大
  学), 2009年5月 27---29日,2009,荷兰。

\item 瑞典中央银行会议 ``Modeling and Forecasting Economic and
  Financial Time Series with State Space
  models'', 2008年10月 17---18日,瑞典斯德哥尔摩。
\end{itemize}

\section*{计算机水平}
\begin{itemize}

\item 掌握基于Hadoop/Spark的大数据计算平台及其统计方法开发。

\item  精通 \textbf{R},Matlab熟练使用SAS,Julia和Python。

\item  掌握GNU/Linux服务器管理,有大型 Linux 集群使用经验。

\item  熟练使用C/C++。
\end{itemize}

\section*{获奖情况}
\begin{itemize}

\item 第二届全国高校经管类实验教学案例大赛二等奖,2017年12月。
\item The 2014 Cramér Prize,瑞典统计学会,2014年3月。

\item 国际贝叶斯协会青年旅行奖励,2012年6月。

\item 瑞典 Knut and Alice Wallenberg 基金会奖励, 2011年8月。

\item 北京市级优秀毕业生, 2007年7月。
\end{itemize}

\section*{担任期刊审稿}
\begin{itemize}
\item \emph{Pattern Recognition}
\item \emph{Neurocomputing}
\item  \emph{Journal of Business and Economics Statistics}
\item  \emph{International Journal of Forecasting}
\item  \emph{Computational Statistics and Data Analysis}
\item  \emph{Journal of Statistical Computation and Simulation}
\item  \emph{Australian \& New Zealand Journal of Statistics}
\item  \emph{Journal of Official Statistics}
\item  \emph{Quantitative Finance}
\item  \emph{Information Sciences}
\item  \emph{Studies in Nonlinear Dynamics \& Econometrics}
\end{itemize}

\section*{其他学术兼职}
\begin{tabular}{l p{0.8\textwidth} l}
  2018--     & 全国工业统计学教学研究会常务理事、中国青年统计学家协会副秘书长 \\
  2014--     & 中国统计教育学会高等教育分会副秘书长。                         \\
  2013--2014 & 2014年金融工程与风险管理国际研讨会执行秘书。                   \\

\end{tabular}

% \section*{联系人}
% \begin{itemize}
% \item Mattias Villani教授, 瑞典林雪平大学计算机系。电子邮件:
%   \url{mattias.villani@liu.se}。
% \item Fan Yang Wallentin 教授, 瑞典乌普萨拉大学统计系。电子邮件:
%   \url{fan.yang@statistik.uu.se}。
% \item Robert Kohn教授,澳洲社会科学院院士, 澳大利亚新南威尔士大学澳大利亚商学
%   院。电子邮件:\url{r.kohn@unsw.edu.au}。
% \end{itemize}

% \footnotesize{\emph{* 联系人联系方式可以单独索取。}}
\end{document}
